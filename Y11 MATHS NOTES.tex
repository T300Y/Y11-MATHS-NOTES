\documentclass{report}

\input{preamble}
\input{macros}
\input{letterfonts}
\usepackage{fancyhdr}
\pagestyle{fancy}
\fancyfoot[RE,LO]{\copyright Jackson 2022}
\graphicspath{ {images/} }
\title{\Huge{Maths}\\Year 11 Notes}
\author{\huge{Jackson Love}}
\date{2022}
\begin{document}

\maketitle
\newpage% or \cleardoublepage
% \pdfbookmark[<level>]{<title>}{<dest>}
\pdfbookmark[section]{\contentsname}{toc}
\tableofcontents
\pagebreak
\newpage
\section{Algebraic Techniques}
\subsection{Simplifying Algebraic expressions}
\thm{Simplifying Algebraic expressions}{When you add \& subtract in algebra you can only combine like terms\newline Questions in Fitzgeralds 1.1}
\qs{}{$$5x + 2y-3 -(x-7y+9)$$}
$$= 5x + 2y -3 -x + 7y -9$$
$$= 4x + 9y -12$$
\qs{}{$$3x(x+2) -4(x-1)$$}
$$=3x^{2}+6x-4x+5$$
$$=3x^{2}-2x + 5$$

\subsection{Substitution in Formulae}
\thm{Substituition in Formulae}{Substituition occurs when you substitute values into an algebraic equation and/or rearrange the equations to make a variable the subject \newline More Questions in 1.2 Fitzgeralds textbook}
\qs{}{\begin{center} If $S = \frac{a(r^{3}-1)}{r-1}$ find S when $a = 5$, $r = 3$\end{center}}
$$= \frac{5(3^{3}-1)}{3-1}$$
$$= \frac{5 \times 26}{2}$$
$$= 5 \times 13$$
$$=65$$
\qs{}{\begin{center} If $A = P(1+\frac{r}{100})^{n}$, find A when $P = 1000$, $r = 10$, $n = 2$\end{center}}
$$= 1000(1+ \frac{10}{100})^{2}$$
$$= 1000 \times 1.21$$
$$= 1210$$
\subsection{Basic Polynomials}
\thm{Basic Polynomials}{There are different types of polynomials include monomial(one term), binomial(two terms) and 
trinomial(three terms)\newline 
Rules for expanding polynomials:\newline
Expanding Perfect/Difference squares ($(y+4)^{2}$),  square first and last terms and multiply the first and last terms together. It should for $a^2+2ab +b^{2}$ unless there is a negative between the two expressions in hich case $-2ab$
}
\qs{}{$$(2y + 5)^{2}$$}
$$a^{2}+2ab+b^{2}$$
$$= 2y^{2}+25+20y$$
\qs{}{$$(x+2)(x^2-5x+6)$$}
$$=-x^{3}-5x^{2}+6x+2x^{2}-10x+12$$
$$= x^{3}-3x^{2}$$
$$= x^{3}-3x^{2}-4x+12$$
\subsection{Fatorising The Sum/Difference of Two Cubes }
\thm{Fatorising The Sum/Difference of Two Cubes}{When factoring Two cubes there are two rules to remember\newline
Rule 1: $a^{3}+b^{3} = (a+b)(a^{2}-ab+b^{2})$ \newline
Rule 2: $a^{3}-b^{3} = (a-b)(a^{2}+ab+b^{2})$
\newline
To remember the sings used in the factorisation an acronym is SOAP(SAME, OPPOSITE, ALWAY, POSITIVE)
}
\qs{}{$$a^{3}b - ab^{3}$$}
$$= ab(a-b)(a+b)$$
\qs{}{$$x^{3}-x^{2}y-9x+9y$$}
$$= x^{2}(x-y)-9(x-y)$$
$$=(x-y)(x^{2}-9)$$
$$= (x-y)(x+3)(x-3)$$
\qs{}{$$(x+5)^{3}+(x-2)^{3}$$}
$$=(2x+3)((x+5)^{2}-((x+5)(x-2))+ (x-2)^{2})$$
$$=(2x+3)(x^{2}+10x+25-x^{2}+2x-5x+10+x^{2}-2x-2x+4)$$
$$=(2x+3)(x^{2}+10x+35-x^{2}+2x-5x+x^{2}-2x-2x+4)$$
$$=(2x+3)(x^{2}+10x+35+2x-5x-2x-2x+4)$$
$$=(2x+3)(x^{2}+3x+35+4)$$
$$=(2x+3)(x^{2}+3x+39)$$
\begin{note}    
    Remember to use FOIL(First, Outside,.Inside Last) to expand brackets
\end{note}
\subsection{Simplifying Algebraic Fractions}
\thm{Simplifying Algebraic Fractions}{When simplifying algebraic fractions it is important to use these two steps: \newline
1. Factorise the numerator and denominator\newline
2. After factorising you can cancel any common factors}
\qs{}{$$\frac{8x^{2} +4x+2}{8x^{3}-1}$$}
$$= \frac{2(4x^{2}+2x+1)}{(2x-1)((2x)^{2}+ (2x \times 1) + 1^{2})}$$
$$= \frac{2(4x^{2}+2x+1)}{(2x-1)(4x^{2}+2x+1)}$$
$$= \frac{2}{2x-1}$$
\qs{}{$$\frac{(x+h)^{3}-x{3}}{h}$$}
$$=\frac{(x+h-x)((x+h)^2+x(x+h)+x^{2})}{h}$$
$$=\frac{h(x^2+2xh+h^{2}+x^{2}+xh+x^{2})}{h}$$
$$=\frac{h(3x^{2}+3xh+h^{2})}{h}$$
$$=3x^{2}+3xh+h^{2}$$
\subsection{Adding \& Subtracting Algebraic Fractions}
\thm{Adding \& Subtracting Algebraic Fractions}{To Add or Subtract Algebraic fractions there are three important steps you need to follow\newline
Rule 1: Factorise all fractions on the numerator \& denominator
Rule 2: Find and create a common denomitor for all fractions(remember to not repeat the same expression more than once)\newline
Rule 3:Simplifying the fraction using like terms}
\qs{}{$$\frac{5}{2a+6}+\frac{a}{a^{2}-9}$$}
$$=\frac{5}{2(a+3)}+\frac{a}{(a+3)(a-3)}$$
$$=\frac{5(a-3)+2a}{2(a+3)(a-3)}$$
$$=\frac{5a-15+2a}{2(a+3)(a-3)}$$
$$=\frac{7a-15}{2(a+3)(a-3)}$$
\qs{}{$$\frac{6}{3x-2}-\frac{8}{4x+1}$$}
$$=\frac{6(4x+1)-8(3x-2)}{(4x+1)(3x-2)}$$
$$=\frac{24x +6 - 24x +16}{(4x+1)(3x-2)}$$
$$=\frac{22}{(4x+1)(3x-2)}$$
\subsection{Surds}
\thm{Rationalising the denominator}{
    Rationalising the denominator involves multiplying the entire fraction 
    by the surd, denominator to rationalise it to a whole number
    \newline
    If the denominator is a binomial and has both a rational and rational ;portion you will need to use the conjugate,the conjugate is the denominator with opposite signs.
    \newline
    if $\frac{1}{3 + \sqrt{2}}$ is the fraction, the conjugate is $3-\sqrt{2}$ as this results in the difference of squares

}
\qs{}{$$\frac{2\sqrt{6}}{5\sqrt{2}}$$}
$$\frac{2\sqrt{6}}{5\sqrt{2}} \times \frac{\sqrt{2}}{\sqrt{2}}$$
$$\frac{2\sqrt{12}}{10}$$
$$\frac{4\sqrt{3}}{10}$$
$$\frac{2\sqrt{3}}{5}$$
\qs{}{$$\frac{1}{3\sqrt{3}+4}$$}
$$\frac{1}{\sqrt{3+4}} \times \frac{\sqrt{3}-4}{\sqrt{3}-4}$$
$$\frac{\sqrt{3}-4}{3-16}$$
$$-\frac{\sqrt{3}-4}{13}$$
\subsection{Completing the square}
\thm{Completing the Square}{

To complete the square with monic quadratics  $x^{2} + bx$ , add $\left(\frac{b}{2}\right)^{2}$ to both sides of the equation
\newline
$x^{2} + bx + \left(\frac{b}{2}\right)^{2} = \left(\frac{b}{2}\right)^{2}$
then solve for $x$
When wanting to complete the square for non-monic quadratics you first must make the equation monic by diving the equation by $a$
$ax^{2}+bx+c=0$
\newline
$x^2+\frac{b}{a}x=-\frac{c}{a}$
\newline
$x^2+\frac{b}{a}x+\left(\frac{b}{2a}\right)^{2}=-\frac{c}{a}+\left(\frac{b}{2a}\right)^{2}$
\newline
$(x+\frac{b}{2a})^{2}=-\frac{c}{a}+\left(\frac{b}{2a}\right)^{2}$
\newline
Then solve like a normal completing the square, the non monic completing the square formula is also how the quadratic formula is derived


}
\qs{}{$$2x^{2}+6x-5=0$$}
$$x^{2}+3x+\frac{9}{4}=\frac{5}{2}+\frac{9}{4}$$
$$(x+\frac{3}{2})^{2}=\frac{19}{4}$$
$$x+\frac{3}{2}=\frac{\pm\sqrt{19}}{2}$$
$$x=\frac{-3\pm\sqrt{19}}{2}$$
\qs{}{$$3x^{2}-5x-1=0$$}
$$x^2-\frac{5}{3}x+\left(-\frac{5}{6}\right)^{2}=\frac{1}{3}+\left(-\frac{5}{6}\right)^{2}$$
$$\left(x-\frac{5}{6}\right)^{2}=\frac{37}{36}$$
$$x-\frac{5}{6}=\frac{\pm\sqrt{37}}{6}$$
$$x=\frac{5\pm\sqrt{37}}{6}$$
\subsection{Indices}
 \thm{Indices}{
Index Laws:\newline
$a^{m}\times a^{n}= a^{n+m}$\newline
$a^{m} \div  a^{n}= a^{m-n}$\newline
$(a^{m})^{n}= a^{nm}$\newline
$(ab)^{n}=a^{n}b^{n}$\newline
$\left(\frac{a}{b}\right)^{n}=\frac{a^{n}}{b^{n}}$\newline
NEGATIVE INDICES:\newline
$x^{-n}=\frac{1}{x^{n}}$\newline
Fractional Indices:\newline
$a^{\frac{m}{n}}=\sqrt[n]{a^{m}}$
 }

\qs{}{$$\frac{1}{\sqrt[3]{(4x^{2}-1)^{2}}}$$}
$$(4x^{2}-1)^{-\frac{2}{3}}$$
\qs{}{$$\frac{x-5+6x^{-1}}{1-2x^{-1}}$$}
$$\frac{x-5+6x^{-1}}{1-2x^{-1}} \times \frac{x}{x}$$
$$\frac{x^{2}-5x+6x}{x-2x}$$
$$\frac{(x-3)(x-2)}{x(1-2)}$$
\newpage
\section{Functions}
\subsection{Funtions and Relations}
\thm{Functrions and Relations}{

    A relation is a set of ordered pairs where variables are related to each other according to a rules\newline
    A set is a list of numbers, ordered pairs etc
    \newline
    \newline
    Types of Relations:\newline
    One-to-One - every element corresponds to  on element in the other set
    \newline
    One-to-Many - where a element in  Set A corresponds to 2 or more elemnts in Set B
    \newline
    Many-to-One - 2 or more elemnts of Set A correspond with 2 or more elements in Set B
    \newline
    \newline
    Functions:
    Functions are a special typoe of relation where every elemnt of Set A corresponds with a unique element of Set B. 
    In a function the domain is the set of all x values that the function could input, 
    the range in the function is the set of all y values that can be potentialy outputted by the function.
    \newline
    \newline
    Vertical Line Test:
    To determine whether something is a function vs a relation we can use the vertical line test which states that if a 
    line onlycuts the y axis at one point it must represent a function.
    \newline
    Horizontal line test:
    We can use the horizontal line test to determine if a relation is one-one or not, 
    if multiple points lie on the same y coordinate then the function cannot be one-one.
}
\qs{}{\begin{center}Find the Domain and range of the equation $\sqrt{x}$\end{center}}
\begin{center}
    Domain: $x\geq0$
    \newline
    Range: $y\geq0$
\end{center}

\qs{}{\begin{center}Find the domain and range of the equation $2+x^{2}$\end{center}}
\begin{center}
    Domain: $\mathbb{R}$ 
    \newline
    Range: $y\geq2$
\end{center}
\newpage
\subsection{Function \& Interval notation}
\thm{Function and Interval notation}{
    Function notaion:
    \newline   
    With function notation like $f(x)$, f is the name of our function and $x$ inside the brackets is the input of the function
    \newline
    So when $f(x)=2x$ then $f(3)=6$
    \newline
    \newline
    Interval notation:
    \newline
    A closed interval is   when the interval contains all endpoints within it.
    \newline
    Example: $y\geq x\geq b$ or in bracket notation $[y,b]$
    \newline
    \newline
    The open interval:
    \newline
    The open interval occurs when the interval does not contain its endpoints.
    \newline
    Example: $y<x<b$ or in bracket notation $(y,b)$
    \newline
    \newline   
    The closed ray:
    \newline
    The closed ray occurs when x is unbounded inn one direction and contains its endpoint.
    \newline
    Example: $x\geq y$ or in bracket notation $[y,\infty)$
    \newline
    \newline
    The open ray:
    \newline
    The open ray occurs when  x is unbounded and does not contain its endpoint.
    \newline
    Example: $x < y$ or or in bracket notation $(-\infty,y)$

}
\newpage

\subsection{Absolute values}
\thm{Absolute values }{
    Absolute values:
    \newline
    Absolute values are a way or measuring the distance a number is from its origin(0), trhis means that an absolute value will always be positive.
    To denote an absolute values we use the symbols $\left\lvert x \right\rvert $
    \newline
    When you are trying to solve an equation with absolute values, it can be positive or negative. Hint, if there is an equatio in an absolute value do not 
    solve until you get rid of the absolute value.
    \newline
    For an example if $\left\lvert x-b \right\rvert = a$ then $x-b=\pm a$
}
\qs{}{$$Solve:\left\lvert x-2 \right\rvert=3$$}
$$x-2=\pm3$$
$$x=2\pm3$$
$$x=5,-1$$
\qs{}{$$\left\lvert m-5 \right\rvert\geq0$$}
$$0\geq m-5\geq0$$
$$5\geq m \geq 5$$
$$m=5$$
\subsubsection{Odd and Even functions}
\thm{Odd and Even functions}{
    A function/relation is even if when graphed it has  line of symmetry from the y axis 
    \newline
    To determine whether a function is even $f(x) = f(-x)$

    \newline
    A function/relation  is called odd if the point of symmetry in the origin, this means that if rotated $180\deg$ trhe graph remains unchanged
    \newline
    To determine whether a function is odd $f(-x)= -f(x)$
    \newline
    If a function is neither odd nor even  you just use "neither"

}
\qs{}{\begin{center}Determine whether the function is odd,even or neither    $f(x)=\frac{3}{x^{2}-4}$    \end{center}}
$$f(-x)= \frac{3}{(-x)^{2}-4}$$
$$=\frac{3}{x^{2}-4}$$
\begin{center} $\therefore f(x)$ is even function \end{center}
\qs{}{\begin{center}Determine whether the function is odd,even or neither    $$f(x)= \frac{x^{3}}{x^{4}-x^{2}}   $$\end{center}}
$$f(-x)=\frac{(-x)^{3}}{(-x)^{4}-(-x)^{2}}$$
$$=-\frac{x^{3}}{x^{4}-x^{2}}$$
$$-f(x)= \frac{(-x)^{3}}{(-x)^{4}-(-x)^{2}}$$
\begin{center} $\therefore f(x)$ is odd function \end{center}

\subsection{Circles and Semi-Circles}
\thm{Circles and Semi Circles}{
    When graphing a circle the traditional formula is $x^{2}+y^{2}= r^{2}$. Because of this the same y and x coordinate may overlap meanign this is not a function
    \newline
    With these circles the origin/centre will be $(0,0)$\newline
    However there is a second formula called the Central formula which is $(x-h)^{2}+(y-k)^{2}=r^{2}$ where this time the circles origin/centre is $(h,k)$`   \newline
    \newline
    The  formula for a semi circles is $y = \pm\sqrt{r^{2}-x^{2}}$, if $y \geq 0$ the semi circle will be positive along the y axis, however if $y \leq 0$ the semi-circle will be negative along the y axis

    }

\qs{}{\begin{center} Find the radius and centre for $(x-4)^{2}+ (y-5)^{2}=16$\end{center}}

$$Centre = (4,5)$$
$$ Radius = 4$$

\qs{}{\begin{center} Find the radius and centre for $(x-5)^{2}+ (y+6)^{2}=49$\end{center}}

$$Centre = (5,-6)$$
$$radius = 7$$
\newpage
\subsection{Cubic Polynomials}
\thm{Cubic Polynomials}{
    Cubic polymials are polynomials of the third degree, a cubic function has one $y$ intercept but upto three $x$ intercepts
    \begin{center}
        \includegraphics[scale=0.25]{hyperbolicfunction1.png}
    \end{center}
    -    in $f(x) = x^{3} + 1$ the constant 1 is the y intercept
    \newline
    If the form is $kx^{3}$ when $k>0$ it is a increasing function and when $k<0$ it is a decreassing function.
    The point of inflection in the $x^{3}$ function is wear the gradient of the line changes
    \newline
    To find the point of inflexion we can get the equation in the form $f(x) = k(x-b)^{3}+c$ which is the same as $kx^{3}$ however the inflexion point is $(b,c)$
    \newline
    The final way a cubic function can be displayed is of $ f(x) = k(x-a)(x-b)(x-c)$


}
\qs{}{\begin{center} Find inflexion point of the equation $2(x-1)^{3}-16$ \end{center} }
$$=k(x-b)^{3} + c$$
$$POI = (b,c)$$
$$POI=(1,16)$$
\qs{}{\begin{center} Find inflexion point of the equation $(x+2)^{3}+8=0$ \end{center} }
$$=k(x-b)^{3} + c$$
$$POI = (b,c)$$
$$POI = (-2,8)$$
\begin{note} 
    As the form is -b, the b value for POI is negative rather than positive. If the b vlue was negative it the POI b would be positive.
\end{note}

\subsection{Hyperbolic Function(Inverse variation)}
\thm{Hyperbolic Function}{
    A hyperbola occurs in the form of $\frac{k}{x}$ where k is a constant. This equation tells us that when one variable increases the other decreases..
    \newline
    A hyperbola is a discontinuous function meaning there are gaps. There are two asymptotes in a hyperbolic function one on the $x$ and one on the $y$ axis. 
    \begin{center}
        \includegraphics[scale=0.5]{hyperbolafunction1.png}
    \end{center}
    To find the y intercept set x = 0 and solve for y
    \newline
    To find the x asymptote  solve expression on the denominator for x.
    \newline
    To find the y asymptote  substitute the x asymtote into the denominator expression and solve for y
        

}

\qs{}{\begin{center} Find the asymptotes and the Domain and Range of the following function $f(x) = \frac{3}{x-3} \end{center}$ }
$$x-3 \neq 0$$
$$x \neq 3$$
$$y \neq 0$$
\begin{center} y int $\neq\frac{3}{0-3}$\end{center}
\begin{center} y int $\neq-1$\end{center}
\begin{center}
    $\therefore$ x asymptote $= 3$, y asymptote $= 0$, y int $= 0$
\end{center}
\begin{center}   
    Domain:
    \newline
    $(-\infty, 3)\cup(3,\infty) $
    \newline
    Range:
    \newline
    $(-\infty,0)\cup(0,\infty)$
\end{center}
\newpage
\qs{}{\begin{center} Find the x and y asymptotes and the y intercept = for the following function $f(x)= -\frac{1}{2x+4}$   \end{center}}
\begin{center}
    $2x+4 \neq 0$
    \newline
    $2x \neq -4$
    \newline
    $x \neq -2$
    \newline
    x asymp $= -2$
    \newline
    y int $= -\frac{1}{0+4}$
    \newline
    y int $= -\frac{1}{4}$
    \newline
    y asymp $= 0$

\end{center}


\subsection{Linear functions}
\subsubsection{Intersection of Two Lines}
\thm{Intersection of two lines}{
    To find the point where two lines intersect we need to solve the two equations simultaneously
}
\qs{}{\begin{center} Find the intersection points of the equations $4x+2y+2=0$ and $3x+5y-9=0$\end{center}}
$$4x+2y+2=0...1$$
$$3x+5y-9=0...2$$
$$...1 \times 3$$
$$12x+6y+6 = 0...3$$
$$...2 \times 4$$
$$12x+20y-36=0...4$$
$$...4 - ...3$$
$$14y - 42 = 0$$
$$y = 3$$
$$4x+6+2 = 0$$
$$x = -2$$
\begin{center}
    $\therefore$ the intersection point of the two lines is $(-2,3)$
\end{center}

\qs{}{Find the equuation of the line above if it passes through points $(4, -2)$}

$$m = \frac{3+2}{-2-4}$$
$$m = -\frac{5}{6}$$
$$y - 3 = -\frac{5}{6}(x+2)$$
$$y-3 = -\frac{5}{6}x + -\frac{10}{6}$$
$$6y-18 = -5x -10$$
$$5x+6y-8 =0$$
\begin{note}
    Two straight lines have three possible variations with them intersecting at the same point, being parallel or coinciding(infinite solutions)
\end{note}
\subsubsection{Solving Simultaneous equations using the k method}
\thm{K method}{
    The k method works for when you need to find the equation of a line that passes through a point aswell as the intersection point of two lines. The formula is $y + k(x) = 0$
    Where $y$ is the first equation and  $x$ is the second equation, It is worth noting that each equations has to equal zero before using the k method. You then subsitute the point values in the equation and solve for k.
    Once you solve for k you subsitute into the first form of the k equation and simplify.
}
\qs{}{\begin{center} Two equations $4x+2y+2=0$ and $3x+5y-9=0$ intersect at $(4,-2)$ Find the equation using the k method   \end{center}}
    $$4x+2y+2+k(3x+5y-9)=0$$
    $$16-4+2+k(12-10-9)=0$$
    $$14-7k=0$$
    $$7k=14$$
    $$k=2$$
    $$4x+2y+2+2(3x+5y-9)=0$$
    $$4x+2y+2+6x+10y-18=0$$
    $$10x+12y-16=0$$
    $$5x+6y-8=0$$
\subsection{Composite functions}
\thm{Composite functions}{
    Composite functions are functons when the output of one is used as the input for another eg. $g(f(x))$. In other words you combine two functions to create a third function with a different output.
}

\qs{}{$$f(x) = 3x \ g(x)= x+4 find \ g(f(x))$$}
$$g(f(x))= 3x+4$$

\qs{}{$$f(x)= \frac{x}{3} \ g(x) = \frac{3}{x} \ find \ g(f(x))$$}
$$g(f(x))= x$$

\subsection{Simultaneous equations of the 2nd degree}
\thm{quadratic simultaneous equations}{
    When dealing with second degreee simultaneous equations  generally you will have two sets of coordinates. \newline
    To solve we first either use the subsitute or elimination method, we then factor and solve as usual.
    \newline
        The reason why we solve these is to find where these two equations intersect eachother.
}
\qs{}{\begin{center}Solve simultaneously $y=x-1$ $y=x^{2}+4x+1$ \end{center}}
$$y = x-1...1$$
$$y=x^{2}+4x+1...2$$
\begin{center}
    sub 1 in 2
\end{center}
$$x-1 = x^{2}+4x+1$$
$$x^{2}+3x+2=0$$
$$(x+2)(x+1)$$
$$x = -2,-1$$
$$y = -3, -2$$
$$\therefore POI:(-2,-3), (-1,-2)$$
\qs{}{\begin{center}Solve simultaneously $y-2x+1=0$ $3y^{2}-y-2x^{2}=0$ \end{center}}
$$y=2x-1...1$$
$$3y^{2}-y-2x^{2}=0...2$$
\begin{center}
    Sub 1 in 2
\end{center}
$$3(2x-1)^{2}-(2x-1)-2x^{2}=0$$
$$3(4x^{2}-4x+1)-2x+1-2x^{2}=0$$
$$12x^{2}-12x+3-2x+1-2x^{2}$$
$$10x^{2}-14x+4=0$$
$$(10x-4)(10x-10)=0$$
$$(5x-2)(x-1)=0$$
$$x= \frac{5}{2},1$$
$$y=2-1$$
$$y=1$$
$$y = 2 \times \frac{2}{5}-1$$
$$y= -\frac{1}{5}$$
$$y = 1, -\frac{1}{5}$$
$$\therefore (\frac{2}{5}, \frac{2}{5}), (1,1)$$
\newline
\subsection{Non-Linear Simultaneous equations}
\thm{Non-Linear Simultaneous equations}{
    To solve non-linear simultaneous equations, you first want to make a variable the subject of one equation to subsitute or eliminate one variable by another operation. After this is complete you can solve like a normal simultaneous equation
}
\qs{}{\begin{center}Solve simultaneously $x^{2}+y^{2}=25$ $x+y=5$ \end{center}}
$$x = 5-y...1$$
$$x^{2}-y^{2}=25...2$$
\begin{center}
    Sub 1 in 2
\end{center}
$$(5-y)^{2}+y^{2}=25$$
$$25-10y+2y^{2}=25$$
$$-10y+2y^{2}=0$$
$$2y(y-5)=0$$
$$y= 0,5$$
$$x=5,0$$
$$\therefore (0,5), (5,0)$$

\qs{}{\begin{center}Solve simultaneously $x+2y=-8$ $xy=8$ \end{center}}
$$x=-2y-8...1$$
$$xy=8...2$$
\begin{center}
    Sub 1 in 2
\end{center}
$$y(-2y-8)=8$$
$$2y^{2}+8y+8=0$$
$$(2y+4)^{2}=0$$
$$(y+2)^{2}=0$$
$$y=-2$$
$$x=-4$$
$$\therefore(-4,-2)$$
\newpage
\subsection{Quadratic functions}
\thm{Quadratic function}{Quadratic functuion are polynomials of the 2nd degree $ax^{2}+bx+c$, these equations form parabolas \newline }
\subsubsection{Turning point}
\thm{Turning point}{
Minimum and maximum values:
\newline
The minimum or maximum value of a parabola defines where its turning point will be \newline
If $a >0$ we use call it the minimum value and if $a<0$ we call it the maximum value
\newline
Turning point:
\newline
The turning point of a parabola lies on the axis of symmetry, the axis of sym is halfway between the x intercepts.\newline
To find the axis of symmetry we use $x = \frac{-b}{2a}$

Once we know the value of the axis of symmetry we then subsitute  its value as x. After the equation is solved the answer is our y coordinate.
\newline
Here is an equation for the value of the coordinate pair for the turning point $(\frac{-b}{2a}, a(\frac{-b}{2a})^{2}+ b(\frac{-b}{2a})+ c)$
}
\qs{}{\begin{center}  Find the turning point of the equation $f(x) = x^2-5x+1$  \end{center}}
$$x = \frac{5}{2}$$
$$y = \left(\frac{5}{2}\right)^{2}-5\left(\frac{5}{2}\right)+1$$
$$y = \frac{25}{4} -\frac{25}{2}+1$$
$$y = -\frac{25}{4}+1$$
$$y = -\frac{21}{4}$$
$$y = -5\frac{1}{4}$$
$$\therefore minval=(2\frac{1}{2}, -5\frac{1}{4})$$
\qs{}{\begin{center}  Find the turning point of the equation $f(x) = -3x^{2}+x-5$  \end{center}}
$$x = \frac{-1}{-6}$$
$$x=\frac{1}{6}$$
$$y = -3(\frac{1}{6})^{2}+\frac{1}{6}-5$$
$$y= -\frac{1}{12}+\frac{2}{12}-5$$
$$y = -\frac{59}{12}$$
$$y = -4\frac{11}{12}$$
$$\therefore max val (\frac{1}{6}, -4\frac{11}{12})$$

\subsubsection{The discriminant }
\thm{The discriminant}{
    The discriminant gives information about the roots of a quadratic equation. The discriminant is $\Delta = b^{2}-4ac$
    \newline
    If the $\Delta > 0$ then  there are two real roots(two x intercepts), if the discriminant is a square number these roots are rational and if the discriminant is not a square then the roots are irrational.
    \newline
    If $\Delta = 0$ there are 2 equal rational roots(one x intercept)
    \newline
    If $\Delta < 0$ there are no real roots(no x intercept)
    \newline\begin{center}
    \includegraphics[scale=0.5]{roots.png}
    \end{center}
}

\qs{}{\begin{center} Find k for which $5x^{2}-2x+k=0$ has real roots and $\Delta \geq 0$\end{center}}
$$a > 0$$
$$4-4(5\times k) \geq 0$$
$$4-20k \geq 0$$
$$-20k \geq -4$$
$$k \leq \frac{1}{5}$$
$$k = \frac{1}{5}$$
\begin{note}
    if $a<0$ we would use a $\leq$ symbol instead of $\geq$
\end{note}
\qs{}{\begin{center}Show that $f(x) = x^{2}-x-2$ has 2 real roots\end{center}}
$$\Delta = 1-4(1 \times -2)$$
$$= 1+8$$
$$=9$$
\begin{center}
    $\therefore \Delta > 0$ and is a perfect square so $x^{2}-x-2$ has two real rational roots
\end{center}
\subsubsection{Finding a quadratic equation}
\thm{Finding a quadratic equation}{
    To find a quadratic equation we use subsitute the coordinate pairs and solve like a simultaneous equation
}
\qs{}{\begin{center} Find the equation that passes $(-1,-3)(0,3),(2,21)$\end{center}}
$$y = ax^{2}+bx+c=$$
$$-3 = a-b+c...1$$
$$3 = c...2$$
$$21=4a+2b+c...3$$

\begin{center}
    Sub 2 in 1
\end{center}
$$a-b=-6...4$$
\begin{center}  
Sub 2 in 3
\end{center}
$$4a+2b+3=21...5$$
\begin{center}
    From 5
\end{center}
$$4a+2b=18$$
$$2a+b=9...6$$
\begin{center}
    Add 6 and 7
\end{center}
$$3a=3$$
$$a = 1$$
$$1-b=-6$$
$$b = 7$$
$$\therefore a = 1, b=7, c =3$$
\newpage

\subsubsection{Solution Set of simultaneous equations}
\thm{Solution Set of simultaneous equations}{

A solution set of simultaneous equations is where you are given two equations ad  are to find the coordinates where the two lines touch, intersect, not intersect each other.\newline
To do this we use the discriminant
For an example, once we find the point of intersection line via subsituition we can find where the two lines touch by using the $\Delta$ and the values of the POI line.\newline
We can find the intersection points by using $b^{2}-4ac > 0$ on the new POI line
\newline
We can also find where the line do not inteersect by using $ b^{2}-4ac < 0$

\begin{note}
    To find these  areas, we use ineqwualities and factorisation. Before we complete the steps above we first need to factorise the POI line and use the x values to test the discriminant inequalities as seen below.
\end{note}
}

\qs{}{\begin{center} For what values of $m$ does line $y=mx-6$ algebraic) touch b)intersect c) not intersect the parabola y+x^{2}-2x+3 \end{center}}
\begin{center}
Point of intersection:

\end{center}
$$4x^{2}-2x+3=mx-6$$
$$x^{2}-2x-mx+9$$
$$x^{2}-(m+2)x+9$$

\begin{center}
    Touch when  $\Delta = 0$

\end{center}
$$(m+2)^{2}-4(1\times 9) = 0 $$
$$m^{2}-4m+32=0$$
$$(m+8)(m-4)= 0$$
\begin{center}
    $\therefore m = 4,-8$ when the two lines touch
\end{center}
\begin{note}
    Remember to use the point of interesection equation when subsituting values for the discriminant
\end{note}
\begin{center}
    Intersect when $\Delta >0$
\end{center}
$$(m+8)(m-4)> 0$$
\begin{center}
    We now need to look at for what numbers satisfy m in this inequality. So we plot -8 and 4 on a number line and see if a value less than -8 can satify the inequality then inbetweeen -8 and 4 and then greter than 4
\end{center}
$$(-9+8)(-9-4)>0$$
\begin{center}
    This is true
\end{center}
$$(-2+8)(2-4)>0$$
\begin{center}
    This is false
\end{center}
$$(5+8)(5-4)>0$$
\begin{center}
    This is true
\end{center}

\begin{center}
    $\therefore m<-8, m>4$
\end{center}

\begin{center}
    Not intersect $\Delta < 0$
\end{center}
$$(m+8)(m-4)<0$$
\begin{center}
    Like before we neeed to plot  -8 and 4 ona number line and look for values which satify the inequality
\end{center}
$$(-9+8)(-9-4)<0$$
\begin{center}
    False
\end{center}
$$(-2+8)(-2-4)$$
\begin{center}
    True
\end{center}
$$(5+8)(5-4)$$
\begin{center}
    False
\end{center}
\begin{center}
    $\therefore -8<m<4$
\end{center}
\subsubsection{Simultaneous equations with three unknowns}
\thm{Simultaneous equations with three unknowns}{
    When dealing with three unknown variables we first have to eliminate a variable and tranistion into a two variable simultaneous equation. Once two variables have been found you can easily find the third.
    \newline
    You must note that there are multiple ways to solve these equations correctly aswell as the importance of numbering the equations

}
\qs{}{\begin{center}Solve: \newline $$a-b+c=7...1$$ $$a+2b-c=-4...2$$  $$3a-b-c=3...3$$ \end{center}}
\begin{center}
    We can first add 1 and 2 as we notice this will eliminate the c variable
\end{center}
$$2a+b=3...4$$
\begin{center}
    We can also add 1 and 3 as we notice thisa will also eliminate the c  variable
\end{center}
$$4a-2b=10$$
$$2a-b=5...5$$
\begin{center}
    We can also notice that if we add 4 and 5 we eliminate the b variable
\end{center}
$$4a=8$$
$$a=2$$
\begin{center}
    Sub $a$ back into 4
\end{center}
$$4+b=3$$
$$b=-1$$
\begin{center}
    We can then subsitute both $a$ and $b$ back into equation 1
\end{center}
$$2+1+c=7$$
$$c=4$$
$$\therefore a =2, b = -1,c =4$$
\newpage
\section{Further work with Functions \textbf{(Extension 1)}}
\subsection{Quadratic inequalities}
\thm{Quadratic inequalities}{
A quadratic inequality is where there is a polynomial of the second degreee in an inequality.
\newline
To solve a quadratic inequality you first want to move everything to the lefthand side of the inequality then factor the quadratic equation.
\newline
After factoring, you use  the solution to find the bounds of the inequality. To do this you plot them on a numberline and check each area to see if it fits the inequality. For an example if the solutions after factorsing were 3 and 7, you would check a number $ <3$ in between 3 and 7 and a number $>7$.
\newline

}
\qs{}{\begin{center} Solve: $$x^{2}+x-6>0$$\end{center}}
$$(x+3)(x-2)>0$$
\begin{center}
    We now check each are to see what values fit the inequality of $<-3$ between -3 and 2 and $>2$
\end{center}
$$(-4+3)(-4-2)>0$$
\begin{center}
    True
\end{center}
$$(-1+3)(-1-2)>0$$
\begin{center}
    False
\end{center}
$$(3+3)(3-2)>0$$
\begin{center}
    True
\end{center}
$$\therefore x<-3 or x>2$$
\qs{}{\begin{center} Solve: $$(2-x)(x-5)(x+1)>0$$ \end{center}}
\begin{center}
    The roots are 2, 5, -1. We now need to check numbers both less than or greater than each of these roots
\end{center}
$$(2+2)(-2-5)(-2+1)>0$$
\begin{center}
    True
\end{center}
$$(2+0)(0+-5)(0+1)>0$$
\begin{center}
    False
\end{center}
$$(2-4)(4-5)(4+1)>0$$
\begin{center}
    True
\end{center}
$$(2-6)(6-5)(6+1)>0$$
\begin{center}
    False
\end{center}
$$\therefore x<-1, 2<x<5$$

\qs{}{\begin{center} Solve: $$2^{2x}-5(2^{x})+4 \leq0$$\end{center}}
\begin{center}
    First we can subsitute $2^{x}$ as $m$
\end{center}
$$m^{2}-5m+4 \leq 0$$
$$(m-4)(m-1)$$
$$ 1 \leqm \leq 4$$
$$2^{x}=4$$
$$x=2$$     
$$2^{x} = 1$$
$$x = 0$$
$$\therefore 0\leq x\leq2$$
\newpage
\subsection{Inequalities with unknown denominator}
\thm{Inequalities with unknown denominator}{
When solving inequalities with an unknown denominator there are two methods, the critical values method and the multily by the square denominator method.\newline
However they both start by finding what $x$ cannot be
\newline
Critical Values method:
\newline
After finding out what $x$ cannot be you solve the inequality by changing the sign to an $=$. Following this step you plot what x cannot be and the solved value for x and test each area/range(like with quadratic inequalities) to find what bounds work.
\newline
After this is completre you can format the inequality

\newline
Multiply by Square denominator method:
\newline
After finding what value x cannot be you muliply each side by the denominator squared, you then move all componnts of the inequality to make one side equal to zero. Following this you factorise this quadratic equation and find the solution for x. You then complete the numberline method  as you do with quadratic inequalities 
}
\qs{}{\begin{center}Solve $$\frac{y^{2}-6}{y} \leq 1$$ \end{center}}
$$y \neq 0$$
$$\frac{y^{2}-6}{y} \times y^{2} \leq 1 \times y^{2}$$
$$y^{3}-6y \leq y^{2}$$
$$y(y^{2}-6) \leq y^{2}$$
$$y(y^{2}-6) +y^{2} \leq 0$$
$$y(y^{2}-y-6) \leq 0$$
$$y(y-3)(y+2)\leq 0$$
\begin{center}
    The roots are  3 and -2
    \newline
    We now use the numberline method and test the numbers in original equation
\end{center}
$$\frac{-3{2}-6}{-3} \leq 1$$
\begin{center}
    True
\end{center}
$$\frac{-1^{2}-6}{-1} \leq 1$$
\begin{center}
    False
\end{center}
$$\frac{1^{2}-6}{1} \leq 1$$
\begin{center}
    True
\end{center}
$$\frac{4^{2}-6}{4} \leq 1$$
\begin{center}
    False
\end{center}
$$\therefore y \leq 2 , 0<y\leq3$$
\qs{}{\begin{center} Solve $$\frac{6}{x+3} \geq 1$$\end{center}}
$$x \neq -3$$
$$\frac{6}{x+3} \times (x+3)^{2} \geq (x+3)^{2}$$
$$6x+18 \geq x^{2}+6x+9$$
$$x^{2}-9 \leq 0$$
$$(x-3)(x+3)$$
$$\therefore -3 < x \leq 3$$
\begin{note}
    We do not need to use the numberline method as we know that because of the $<$ sign, we are looking for all x values below the x axis.
    \newline 
    If the sign was $>$ we would be looking for all coordinates above the x  axis 
\end{note}

\subsection{Inequalities Involving Absolute values}
\thm{Inequalities involving Absolute values}{
As absolute values can have two inputs(+ or -) for the same one output, we split them up into two simultaneous inequalities to solve.
\newline
Because of this, the negative potential input flips the inequality sign. After splitting up the absolute value into to inequalities you should either have a $<$ and a $>$ or a $\leq$ and $geq$


}
\qs{}{\begin{center} Solve : $$\left\lvert 5b-7 \right\rvert \geq 3 $$\end{center}}
$$5b-7 \geq 3$$
$$5b-7 \leq -3$$
$$5b \geq 10$$
$$b \geq 2$$
$$5b \leq 4$$
$$b \leq \frac{4}{5}$$
$$\therefore b \geq 2, b \leq \frac{4}{5}$$


\qs{}{\begin{center}$\left\lvert 2y-1\right\rvert< 5$  \end{center}}
$$2y-1 < 5$$
$$2y < 6$$
$$y < 3$$
$$2y-1 > -5$$
$$2y > -4$$
$$y > -2$$
$$\therefore y>-2, y < 3$$






\subsection{Inequalities Involving Square roots}
\thm{Inequalities involving Square roots}{

}
\qs{}{\begin{center} \end{center}}
\qs{}{\begin{center} \end{center}}
\end{document}
