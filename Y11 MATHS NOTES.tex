\documentclass{report}

\input{preamble}
\input{macros}
\input{letterfonts}
\usepackage{fancyhdr}
\pagestyle{fancy}
\fancyfoot[RE,LO]{\copyright Jackson 2022}
\graphicspath{ {images/} }
\title{\Huge{Maths}\\Year 11 Notes}
\author{\huge{Jackson Love}}
\date{2022}
\begin{document}

\maketitle
\newpage% or \cleardoublepage
% \pdfbookmark[<level>]{<title>}{<dest>}
\pdfbookmark[section]{\contentsname}{toc}
\tableofcontents
\pagebreak
\newpage
\section{Algebraic Techniques}
\subsection{Simplifying Algebraic expressions}
\thm{Simplifying Algebraic expressions}{When you add \& subtract in algebra you can only combine like terms\newline Questions in Fitzgeralds 1.1}
\qs{}{$$5x + 2y-3 -(x-7y+9)$$}
$$= 5x + 2y -3 -x + 7y -9$$
$$= 4x + 9y -12$$
\qs{}{$$3x(x+2) -4(x-1)$$}
$$=3x^{2}+6x-4x+5$$
$$=3x^{2}-2x + 5$$

\subsection{Substitution in Formulae}
\thm{Substituition in Formulae}{Substituition occurs when you substitute values into an algebraic equation and/or rearrange the equations to make a variable the subject \newline More Questions in 1.2 Fitzgeralds textbook}
\qs{}{\begin{center} If $S = \frac{a(r^{3}-1)}{r-1}$ find S when $a = 5$, $r = 3$\end{center}}
$$= \frac{5(3^{3}-1)}{3-1}$$
$$= \frac{5 \times 26}{2}$$
$$= 5 \times 13$$
$$=65$$
\qs{}{\begin{center} If $A = P(1+\frac{r}{100})^{n}$, find A when $P = 1000$, $r = 10$, $n = 2$\end{center}}
$$= 1000(1+ \frac{10}{100})^{2}$$
$$= 1000 \times 1.21$$
$$= 1210$$
\subsection{Basic Polynomials}
\thm{Basic Polynomials}{There are different types of polynomials include monomial(one term), binomial(two terms) and 
trinomial(three terms)\newline 
Rules for expanding polynomials:\newline
Expanding Perfect/Difference squares ($(y+4)^{2}$),  square first and last terms and multiply the first and last terms together. It should for $a^2+2ab +b^{2}$ unless there is a negative between the two expressions in hich case $-2ab$
}
\qs{}{$$(2y + 5)^{2}$$}
$$a^{2}+2ab+b^{2}$$
$$= 2y^{2}+25+20y$$
\qs{}{$$(x+2)(x^2-5x+6)$$}
$$=-x^{3}-5x^{2}+6x+2x^{2}-10x+12$$
$$= x^{3}-3x^{2}$$
$$= x^{3}-3x^{2}-4x+12$$
\subsection{Fatorising The Sum/Difference of Two Cubes }
\thm{Fatorising The Sum/Difference of Two Cubes}{When factoring Two cubes there are two rules to remember\newline
Rule 1: $a^{3}+b^{3} = (a+b)(a^{2}-ab+b^{2})$ \newline
Rule 2: $a^{3}-b^{3} = (a-b)(a^{2}+ab+b^{2})$
\newline
To remember the sings used in the factorisation an acronym is SOAP(SAME, OPPOSITE, ALWAY, POSITIVE)
}
\qs{}{$$a^{3}b - ab^{3}$$}
$$= ab(a-b)(a+b)$$
\qs{}{$$x^{3}-x^{2}y-9x+9y$$}
$$= x^{2}(x-y)-9(x-y)$$
$$=(x-y)(x^{2}-9)$$
$$= (x-y)(x+3)(x-3)$$
\qs{}{$$(x+5)^{3}+(x-2)^{3}$$}
$$=(2x+3)((x+5)^{2}-((x+5)(x-2))+ (x-2)^{2})$$
$$=(2x+3)(x^{2}+10x+25-x^{2}+2x-5x+10+x^{2}-2x-2x+4)$$
$$=(2x+3)(x^{2}+10x+35-x^{2}+2x-5x+x^{2}-2x-2x+4)$$
$$=(2x+3)(x^{2}+10x+35+2x-5x-2x-2x+4)$$
$$=(2x+3)(x^{2}+3x+35+4)$$
$$=(2x+3)(x^{2}+3x+39)$$
\begin{note}    
    Remember to use FOIL(First, Outside,.Inside Last) to expand brackets
\end{note}
\subsection{Simplifying Algebraic Fractions}
\thm{Simplifying Algebraic Fractions}{When simplifying algebraic fractions it is important to use these two steps: \newline
1. Factorise the numerator and denominator\newline
2. After factorising you can cancel any common factors}
\qs{}{$$\frac{8x^{2} +4x+2}{8x^{3}-1}$$}
$$= \frac{2(4x^{2}+2x+1)}{(2x-1)((2x)^{2}+ (2x \times 1) + 1^{2})}$$
$$= \frac{2(4x^{2}+2x+1)}{(2x-1)(4x^{2}+2x+1)}$$
$$= \frac{2}{2x-1}$$
\qs{}{$$\frac{(x+h)^{3}-x{3}}{h}$$}
$$=\frac{(x+h-x)((x+h)^2+x(x+h)+x^{2})}{h}$$
$$=\frac{h(x^2+2xh+h^{2}+x^{2}+xh+x^{2})}{h}$$
$$=\frac{h(3x^{2}+3xh+h^{2})}{h}$$
$$=3x^{2}+3xh+h^{2}$$
\subsection{Adding \& Subtracting Algebraic Fractions}
\thm{Adding \& Subtracting Algebraic Fractions}{To Add or Subtract Algebraic fractions there are three important steps you need to follow\newline
Rule 1: Factorise all fractions on the numerator \& denominator
Rule 2: Find and create a common denomitor for all fractions(remember to not repeat the same expression more than once)\newline
Rule 3:Simplifying the fraction using like terms}
\qs{}{$$\frac{5}{2a+6}+\frac{a}{a^{2}-9}$$}
$$=\frac{5}{2(a+3)}+\frac{a}{(a+3)(a-3)}$$
$$=\frac{5(a-3)+2a}{2(a+3)(a-3)}$$
$$=\frac{5a-15+2a}{2(a+3)(a-3)}$$
$$=\frac{7a-15}{2(a+3)(a-3)}$$
\qs{}{$$\frac{6}{3x-2}-\frac{8}{4x+1}$$}
$$=\frac{6(4x+1)-8(3x-2)}{(4x+1)(3x-2)}$$
$$=\frac{24x +6 - 24x +16}{(4x+1)(3x-2)}$$
$$=\frac{22}{(4x+1)(3x-2)}$$
\subsection{Surds}
\thm{Rationalising the denominator}{
    Rationalising the denominator involves multiplying the entire fraction 
    by the surd, denominator to rationalise it to a whole number
    \newline
    If the denominator is a binomial and has both a rational and rational ;portion you will need to use the conjugate,the conjugate is the denominator with opposite signs.
    \newline
    if $\frac{1}{3 + \sqrt{2}}$ is the fraction, the conjugate is $3-\sqrt{2}$ as this results in the difference of squares

}
\qs{}{$$\frac{2\sqrt{6}}{5\sqrt{2}}$$}
$$\frac{2\sqrt{6}}{5\sqrt{2}} \times \frac{\sqrt{2}}{\sqrt{2}}$$
$$\frac{2\sqrt{12}}{10}$$
$$\frac{4\sqrt{3}}{10}$$
$$\frac{2\sqrt{3}}{5}$$
\qs{}{$$\frac{1}{3\sqrt{3}+4}$$}
$$\frac{1}{\sqrt{3+4}} \times \frac{\sqrt{3}-4}{\sqrt{3}-4}$$
$$\frac{\sqrt{3}-4}{3-16}$$
$$-\frac{\sqrt{3}-4}{13}$$
 \subsection{Completing the square}
 \thm{Completing the Square}{
 
 }
 
 \subsection{Quadratic Formula}
 \thm{Quadratic Formula}{
 
 }
 \subsection{Indices}
 \thm{Indices}{

}
\end{document}
